%%%%%%%%%%%%%%%%%%%%%%%%%%%%%%%%%%%%%%%%%%%%%%%%%%%%%%%%%%%%%%%%%%%%%%%%
% Template adapted from latex-mimosis:
% https://github.com/Pseudomanifold/latex-mimosis
%
% Copyright (c) 2022 Antonio Coín
%
% This work is licensed under a
% Creative Commons Attribution-ShareAlike 4.0 International License.
%
% You should have received a copy of the license along with this
% work. If not, see <http://creativecommons.org/licenses/by-sa/4.0/>.
%%%%%%%%%%%%%%%%%%%%%%%%%%%%%%%%%%%%%%%%%%%%%%%%%%%%%%%%%%%%%%%%%%%%%%%%


\documentclass{mimosis}

% Relative paths
\newcommand{\thesisPath}{chapters}

% Misc. packages
\usepackage{metalogo}
\usepackage{etoolbox}
\usepackage{tocloft}
\usepackage{cancel}
\usepackage{bbding}

% Licensing
\usepackage[
    type={CC},
    modifier={by-sa},
    version={4.0},
]{doclicense}

% Make chapter number show in arabic
\renewcommand\cftchappresnum{\normalfont}

%% Primary color
\newcommand{\titlecolor}{teal}

% Large number and color for chapter
\renewcommand*{\chapterformat}{%
  \fontsize{60}{65}\selectfont\color{\titlecolor}\(\thechapter\)\autodot\enskip
}

%
% Colour for section numbers
%
\renewcommand*{\sectionformat}{%
  \textcolor{\titlecolor}{\(\thesection\)}\autodot\enskip%
}

%
% Colour for subsection numbers
%
\renewcommand*{\subsectionformat}{%
  \textcolor{\titlecolor}{\(\thesubsection\)}\autodot\enskip%
}

%
% Colour for subsubsection numbers
%
\renewcommand*{\subsubsectionformat}{%
  \textcolor{\titlecolor}{\(\thesubsubsection\)}\autodot\enskip%
}

%%%%%%%%%%%%%%%%%%%%%%%%%%%%%%%%%%%%%%%%%%%%%%%%%%%%%%%%%%%%%%%%%%%%%%%%
% Custom commands
%%%%%%%%%%%%%%%%%%%%%%%%%%%%%%%%%%%%%%%%%%%%%%%%%%%%%%%%%%%%%%%%%%%%%%%%

%% Math operators
\DeclareMathOperator*\argmin{arg\,min}
\DeclareMathOperator\supp{supp}
\DeclareMathOperator\Cov{Cov}
\DeclareMathOperator\Var{Var}

%% Math symbols
\newcommand{\N} {\ensuremath{\mathds{N}}}
\newcommand{\R} {\ensuremath{\mathds{R}}}
\newcommand{\E} {\ensuremath{\mathds{E}}}
\newcommand{\I} {\ensuremath{\mathds{I}}}
\renewcommand{\P} {\ensuremath{\mathds{P}}}

\newcommand{\D} {\ensuremath{\mathcal{D}}}
\newcommand{\X} {\ensuremath{\mathcal{X}}}
\newcommand{\K} {\ensuremath{\mathcal{K}}}
\newcommand{\T} {\ensuremath{\mathcal{T}}}
\newcommand{\U} {\ensuremath{\mathcal{U}}}
\newcommand{\Lcal} {\ensuremath{\mathcal{L}}}
\newcommand{\Hcal} {\ensuremath\mathcal{H}}

%% Better overline
\newcommand{\overbar}[1]{\mkern 1.5mu\overline{\mkern-1.5mu#1\mkern-1.5mu}\mkern 1.5mu}

%% Scalar product
\newcommand\dotprod[2]{\left\langle #1, #2 \right\rangle}

%%%%%%%%%%%%%%%%%%%%%%%%%%%%%%%%%%%%%%%%%%%%%%%%%%%%%%%%%%%%%%%%%%%%%%%%
% Hyperlinks & bookmarks
%%%%%%%%%%%%%%%%%%%%%%%%%%%%%%%%%%%%%%%%%%%%%%%%%%%%%%%%%%%%%%%%%%%%%%%%

\definecolor{webgreen}{rgb}{0,0.5,0}
\usepackage[%
  colorlinks = true,
  citecolor  = teal, %webgreen,
  linkcolor  = teal, % RoyalBlue,
  urlcolor   = RoyalBlue,
  unicode,
  ]{hyperref}

\usepackage{bookmark}

%%%%%%%%%%%%%%%%%%%%%%%%%%%%%%%%%%%%%%%%%%%%%%%%%%%%%%%%%%%%%%%%%%%%%%%%
% Bibliography
%%%%%%%%%%%%%%%%%%%%%%%%%%%%%%%%%%%%%%%%%%%%%%%%%%%%%%%%%%%%%%%%%%%%%%%%

\usepackage[%
  autocite     = plain,
  backend      = biber,
  natbib       = true,
  doi          = false,
  url          = false,
  isbn         = false,
  giveninits   = true,
  hyperref     = true,
  maxbibnames  = 99,
  maxcitenames = 2,
  mincitenames = 1,
  uniquelist   = false,
  dashed       = false,
  sortcites    = true,
  style        = authoryear-comp,
  backref      = true
  ]{biblatex}


% Format references as {last-name}, {first-name-initials}
\DeclareNameAlias{sortname}{family-given}

\addbibresource{bibliography.bib}

%%%
% Math Environments
%%%

\theoremstyle{theorem}
\newtheorem{theorem}{Theorem}[chapter]
\newtheorem{corollary}[theorem]{Corollary}
\newtheorem{lemma}[theorem]{Lemma}
\newtheorem{proposition}[theorem]{Proposition}

\theoremstyle{definition}
\newtheorem{definition}[theorem]{Definition}
\newtheorem{example}[theorem]{Example}

\theoremstyle{remark}
\newtheorem{remark}{Remark}

%%%
% Lists and footnotes
%%%

% Lists
\usepackage[shortlabels]{enumitem}
%\setlist{itemsep=0.05em}
\setlist{noitemsep}
\setlist[1]{labelindent=\parindent} % < Usually a good idea
\setlist[enumerate,1]{label=\((\roman*)\)}

% Continuous counter for footnotes and figures
\counterwithout{footnote}{chapter}
\counterwithout{figure}{chapter}

%%%%%%%%%%%%%%%%%%%%%%%%%%%%%%%%%%%%%%%%%%%%%%%%%%%%%%%%%%%%%%%%%%%%%%%%
% Fonts
%%%%%%%%%%%%%%%%%%%%%%%%%%%%%%%%%%%%%%%%%%%%%%%%%%%%%%%%%%%%%%%%%%%%%%%%

\ifxetexorluatex
  \usepackage{unicode-math}
  \setmainfont{EB Garamond}
  \setmathfont{latinmodern-math.otf}
  \setmathfont[range={\mathcal,\mathbfcal}]{Garamond-Math.otf}
  %\setmathfont{Garamond-Math.otf}
  \setmonofont[Scale=MatchLowercase]{Source Code Pro}
\else
  \usepackage[lf]{ebgaramond}
  \usepackage[oldstyle,scale=0.7]{sourcecodepro}
  \singlespacing
\fi

%%%%%%%%%%%%%%%%%%%%%%%%%%%%%%%%%%%%%%%%%%%%%%%%%%%%%%%%%%%%%%%%%%%%%%%%
% Incipit
%%%%%%%%%%%%%%%%%%%%%%%%%%%%%%%%%%%%%%%%%%%%%%%%%%%%%%%%%%%%%%%%%%%%%%%%

\title{\textbf{Bayesian RKHS-based methods in functional regression}}
\author{Antonio Coín Castro}

\begin{document}

\frontmatter
  % Titlepage

\begin{titlepage}
  \vspace*{5cm}
  \makeatletter
  \begin{center}
    \begin{singlespace*}
    \begin{Huge}
      \@title
    \end{Huge}\\[1cm]
    %
    \begin{large}
      Master's thesis
    \end{large}
    %
    \vskip 0.1cm
    \emph{by}
    \vskip 0.18cm
    \begin{Large}
    \textsc{\@author}
  \end{Large}
  \end{singlespace*}
        %
    \vfill

    Under the supervision of
    \begin{large}
    \vskip 0.1cm
    José\,Ramón\,Berrendero\,Díaz\\
    Antonio\,Cuevas\,González\\[1cm]
  \end{large}

    A document submitted in partial fulfillment
    of the requirements for the\\
    \emph{Master's degree in Data Science}\\
    at\\
    \textsc{Universidad Autónoma de Madrid}\\[1cm]

    September, 2022
  \end{center}
  \makeatother
\end{titlepage}
\newpage


%\thispagestyle{empty}
%\null
%\newpage



\thispagestyle{empty}
%\renewcommand*{\chapterpagestyle}{empty}
\pagebreak
\hspace{0pt}
%\vspace{17cm}
\vfill
\begin{center}
    \begin{minipage}[t]{12.5cm}
        \doclicenseThis
        This document has been typeset using a modification of the \textit{mimosis} template at \url{https://github.com/Pseudomanifold/latex-mimosis}. The \LaTeX\ source code is openly available at \texttt{\url{https://www.github.com/antcc/tfm}}, while the Python source code for the experiments carried out can be found in \texttt{\url{https://www.github.com/antcc/rk-bfr}}.
    \end{minipage}
\end{center}
\vfill
\hspace{0pt}
\pagebreak
\newpage

%\thispagestyle{empty}
%\null
%\newpage

  %%%%%%%%%%%%%%%%%%%%%%%%%%%%%%%%%%%%%%%%%%%%%%%%%%%%%%%%%%%%%%%%%%%%%%%%
% Copyright (c) 2022 Antonio Coín
%
% This work is licensed under a
% Creative Commons Attribution-ShareAlike 4.0 International License.
%
% You should have received a copy of the license along with this
% work. If not, see <http://creativecommons.org/licenses/by-sa/4.0/>.
%%%%%%%%%%%%%%%%%%%%%%%%%%%%%%%%%%%%%%%%%%%%%%%%%%%%%%%%%%%%%%%%%%%%%%%%

\begin{center}
  \begin{Large}
  \textsc{Abstract}
\end{Large}
\end{center}
%
\noindent We propose a novel Bayesian approach for functional linear and logistic regression models based on the theory of Reproducing Kernel Hilbert Spaces (RKHS's). These new models build upon the RKHS associated with the covariance function of the underlying stochastic process, and can be viewed as a finite-dimensional approximation to the classical functional regression paradigm. The corresponding functional model (or the functional logistic equation in the case of binary response) is determined by a function living on a dense subspace of the RKHS of interest. By imposing a suitable prior distribution on this space, we can perform data-driven inference via standard Bayes methodology, and the posterior distribution can be estimated through Markov chain Monte Carlo methods. Several prediction strategies derived from this posterior distribution turn out to be competitive against other usual alternatives in both simulated examples and real data sets, including a Bayesian-motivated variable selection procedure.\\

\noindent
\textsc{keywords:} functional data, linear regression, logistic regression, reproducing kernel Hilbert space, Bayesian inference, Markov chain Monte Carlo.

\incomment{add blank page} 


  \tableofcontents

\mainmatter

%%%%%%%%%%%%%%%%%%%%%%%%%%%%%%%%%%%%%%%%%%%%%%%%%%%%%%%%%%%%%%%%%%%%%%%%
% Copyright (c) 2022 Antonio Coín
%
% This work is licensed under a
% Creative Commons Attribution-ShareAlike 4.0 International License.
%
% You should have received a copy of the license along with this
% work. If not, see <http://creativecommons.org/licenses/by-sa/4.0/>.
%%%%%%%%%%%%%%%%%%%%%%%%%%%%%%%%%%%%%%%%%%%%%%%%%%%%%%%%%%%%%%%%%%%%%%%%

%%%%%%%%%%%%%%%%%%%%%%%%%%%%%%%%%%%%%%%%%%%%%%%%%%%%%%%%%%%%%%%%%%%%%%%%
\chapter{Introduction}\label{ch:introduction}
%%%%%%%%%%%%%%%%%%%%%%%%%%%%%%%%%%%%%%%%%%%%%%%%%%%%%%%%%%%%%%%%%%%%%%%%

Over the last few decades, situations involving data in the form of functions have become commonplace in many statistical scenarios, as more and more information is available worldwide with an ever-increasing level of granularity in the measurements. In particular, functional data problems are far from unheard of in the data science and machine learning community, since they have attracted the attention of researchers and practitioners equally. Medical data, weather indicators or stock exchange indices are examples of elements that benefit from a functional treatment, where the observations are regarded as single entities rather than as a conglomerate of individual points.

Under a functional framework, the objects of interest are \textit{random functions} instead of random points in a finite-dimensional space. While in principle the functional data could be simply regarded as a discretized vector in a very high dimension (an indeed such a discretization is performed in practice), there are often many advantages in taking into account the functional nature of the data, ranging from modeling the possibly high correlation among points that are close in the domain, to extracting information that may be hidden in the derivatives of the function in question. Thus, the general idea is to assume the existence of an underlying sufficiently smooth function that generates each (possibly noisy) functional observation, even though we only record it on a finite grid of points.

To see what this kind of data looks like, Figure~\ref{fig:tecator_orig} shows an example of a functional data set, which is known in the literature as the Tecator data set, and whose elements represent near-infrared absorbance curves of meat samples. The objective here is to predict the fat content based on this absorbance spectrum, separating the samples into those with ``high'' and ``low'' fat content. At first glance it does not seem that the trajectories contain much relevant information to help classify the samples. However, after a suitable smoothing of the data (e.g. by representing each function in a Fourier basis) we can take the derivatives of the curves. In this case, after differentiating twice a clearer pattern emerges (see Figure~\ref{fig:tecator_derivatives}), one from which inference and prediction will surely be easier.

\begin{figure}[ht!]
  \begin{subfigure}[b]{0.48\textwidth}
    \includegraphics[width=\textwidth]{tecator}
    \caption{Original curves}\label{fig:tecator_orig}
  \end{subfigure}
  \hfill
  \begin{subfigure}[b]{0.48\textwidth}
    \includegraphics[width=\textwidth]{tecator_derivative}
    \caption{Second derivatives}\label{fig:tecator_derivatives}
  \end{subfigure}
  \caption{Curves in the Tecator data set and their second derivatives (after smoothing).}\label{fig:tecator}
\end{figure}

In recent years numerous proposals have arisen on how to suitably deal with functional data, all of them encompassed under the term Functional Data Analysis (FDA), which essentially explores statistical techniques to process, model and make inference on data varying over a continuum. A partial survey on such techniques and methods is \citet{cuevas2014partial}, while a more detailed exposition of the theory and applications can be found for example in \citet{ramsay2005functional}, \citet{hsing2015theoretical} or the book by \citet{horvath2012inference}. As the name suggests, FDA techniques are heavily inspired by functional analysis tools and methods: Hilbert spaces, orthonormal systems, linear operators, and so on. In particular, a notion that also intersects with the classical theory of machine learning and pattern recognition, and that has gained traction in recent years, is that of reproducing kernel Hilbert spaces (RKHS's). We will demonstrate throughout this work how these spaces of functions possess properties that allow for an efficient treatment of functional data.

On the other hand, Bayesian inference methods are ubiquitous in the realm of statistics, and their usual non-parametric approach also makes use of random functions, though in a slightly different manner than in the FDA context. However, the two methodologies can certainly interact and benefit from one another, as we intend to show in this thesis. We will be particularly interested in Markov chain Monte Carlo (MCMC) methods, which allow us to approximate an arbitrary posterior distribution through a function proportional to its density.

\begin{center}
\color{teal}\FourStar
\end{center}

In this work we are concerned with functional linear and logistic regression models, that is, situations where the goal is to predict a continuous or dichotomous variable from functional observations. Even though these problems can be formally stated with almost no differences from their finite-dimensional counterparts, there are some fundamental challenges as well as some subtle drawbacks that emerge as a result of working in infinite dimensions. Moreover, we will concentrate our efforts on the case in which the response is a scalar, though function-on-function and function-on-scalar regression are also interesting scenarios widely explored in the literature. To set a common framework, throughout this work we will consider a scalar response variable \(Y\) (either continuous or binary) which has some dependence on a stochastic \(L^2\)-process \(X=X(t)=X(t, \omega)\) with trajectories in \(L^2[0, 1]\) (i.e. a process with finite second moments and whose realizations are square-integrable functions indexed on \([0,1]\)). The underlying probability space \((\Omega, \mathcal A, \P)\) is not important. We will further suppose that \(X\) is centered, that is to say, its mean function \(m(t)=\E[X(t)]\) vanishes for all \(t\in[0,1]\). In addition, we will tacitly assume the existence of a \textit{labeled} data set \(\D_n =\{(x_i, y_i): i=1,\dots, n\}\) of independent observations from \((X, Y)\), where the functional observations are recorded on a common finite grid \(\{t_j\}\subset [0, 1]\). Our ultimate aim will be to accurately predict the response corresponding to unlabeled samples from \(X\). Figure~\ref{fig:linear_data_example} depicts a typical data set used in functional regression, and we already saw in Figure~\ref{fig:tecator} what a functional classification data set may look like.

\enlargethispage{1\baselineskip}

\begin{figure}[htbp!]
  \centering
  \includegraphics[width=0.9\textwidth]{linear_data_example}
  \caption{Simulated data set for functional regression. On the left we have \(n=50\) functional observations on an equispaced grid of \(N=100\) points on \([0,1]\). To each observation corresponds a real number; the distribution of these responses is shown on the right.}\label{fig:linear_data_example}
\end{figure}


\section{Objectives and scope}

We list below the main objectives of this work in no particular order.

\begin{enumerate}[1.]
  \item To perform a brief but thorough literature review that contextualizes functional linear and logistic regression within statistics and machine learning, exploring the predominant models and techniques.
  \item To propose a novel RKHS-based functional model for linear and logistic regression that builds on existing work and focuses on simplicity, both in terms of interpretation and implementation.
  \item To describe and implement a general Bayesian approach for parameter estimation within the suggested model.
  \item To introduce the tools needed to specify the proposed model and put it into practice, mainly reproducing kernel Hilbert spaces and Markov chain Monte Carlo methods.
  \item To carry out an extensive experimental study to test these new models and compare them to existing methods, both in simulations and in real-world scenarios.
\end{enumerate}

It is beyond the scope of this thesis to provide a complete review of FDA as a whole, or to delve too deeply into the details and inner workings of most models and techniques mentioned. Nevertheless, references are often provided throughout the text to point the interested reader towards more specialized resources.


\section{Structure overview}

In Chapter~\ref{ch:background} we summarize the relevant literature related to our problem, along with a review of the basics of RKHS's and MCMC methods. Chapter~\ref{ch:bayesian} is devoted to explaining the Bayesian methodology and the functional regression models we propose. Then, in Chapter~\ref{ch:model-choice} we present a short discussion of theoretical and computational details that have led to the concrete specification of the model, as well as some validation techniques. The empirical results of the experimentation are contained in Chapter~\ref{ch:experiments}. Lastly, the conclusions drawn from this work and future paths of research are reviewed in Chapter~\ref{ch:conclusions}.

%%%%%%%%%%%%%%%%%%%%%%%%%%%%%%%%%%%%%%%%%%%%%%%%%%%%%%%%%%%%%%%%%%%%%%%%
% Copyright (c) 2022 Antonio Coín Castro
%
% This work is licensed under a
% Creative Commons Attribution-ShareAlike 4.0 International License.
%
% You should have received a copy of the license along with this
% work. If not, see <http://creativecommons.org/licenses/by-sa/4.0/>.
%%%%%%%%%%%%%%%%%%%%%%%%%%%%%%%%%%%%%%%%%%%%%%%%%%%%%%%%%%%%%%%%%%%%%%%%

%%%%%%%%%%%%%%%%%%%%%%%%%%%%%%%%%%%%%%%%%%%%%%%%%%%%%%%%%%%%%%%%%%%%%%%%
\chapter{Background and related work}
%%%%%%%%%%%%%%%%%%%%%%%%%%%%%%%%%%%%%%%%%%%%%%%%%%%%%%%%%%%%%%%%%%%%%%%%

\subsubsection{\(L^2\)-models, shortcomings and alternatives}

 The most common functional linear regression model is the classical \(L^2\)-model, widely popularized since the first edition (1997) of the monograph by~\citet{ramsay2005functional}. It can be seen as a generalization of the usual finite-dimensional model, replacing the scalar product in \(\R^d\) for that of the functional space \(L^2[0,1]\):
\begin{equation}\label{eq:l2-linear-model}
Y = \alpha_0 + \dotprod{X}{\beta} + \epsilon = \alpha_0 + \int_0^1 X(t)\beta(t)\, dt + \epsilon,
\end{equation}
where \(\alpha_0\in \R\), \(\epsilon\) is a random error term independent from \(X\) with \(\E [\epsilon]=0\), and the functional slope parameter \(\beta=\beta(\cdot)\) is assumed to be a member of the infinite-dimensional space \(L^2[0, 1]\). In this case, the inference on \(\beta\) is hampered by the fact that \(L^2[0,1]\) is an extremely wide space that also contains many non-smooth or ill-behaved functions, so that any estimation procedure involving optimization on it would typically be hard. In spite of this, model~\eqref{eq:l2-linear-model} is not flexible enough to include ``simple'' finite-dimensional models based on linear combinations of the marginals, such as \(Y=\alpha_0 + \beta_1 X(t_1)+ \cdots + \beta_p X(t_p) + \epsilon\) for some constants \(\beta_j\in\R\) and instants \(t_j\in[0,1]\); see \citet{berrendero2020general} for additional details on this. Moreover, the non-invertibility of the covariance operator associated with \(X\) (defined in Section~\ref{sec:rkhs}), which plays the role of the covariance matrix in the infinite case, invalidates the usual least squares theory. Thus, some regularization or dimensionality reduction technique is needed for parameter estimation.

A similar \(L^2\)-based functional logistic equation can be derived for the binary classification problem via the logistic function:
\begin{equation}\label{eq:l2-logistic-model}
  \Prob(Y=1 \mid X) = \frac{1}{1 + \exp\{-\alpha_0 - \dotprod{X}{\beta}\}},
\end{equation}
where \(\alpha_0 \in \R\) and \(\beta \in L^2[0, 1]\). In this situation, the most common way of estimating the slope function \(\beta\) is via its Maximum Likelihood Estimator (MLE). However, not only do the same complications as in the linear regression model apply in this situation, but there is also the additional problem that in functional settings the MLE does not exist with probability one under fairly general conditions \citep[see][Sec.~3.2]{berrendero2018functional}.

It turns out that in both scenarios a natural alternative to the \(L^2\)-model is the so-called Reproducing Kernel Hilbert Space (RKHS) model, which instead assumes the unknown functional parameter to be a member of the RKHS associated with the covariance function of the process \(X\), making use of the scalar product of that functional space. As we will show later on, not only is this model simpler and arguably easier to interpret, but it also constrains the parameter space to smoother and more manageable functions. In fact, it includes a model based on finite linear combinations of the marginals of \(X\) as a particular case, which is especially appealing to practitioners confronted with functional data problems due to its simplicity. These RKHS-based models and their idiosyncrasies have been explored in \citet{berrendero2019rkhs, berrendero2020general} in the functional linear regression setting, and in \citet{berrendero2018functional, berrendero2018use} for the case of functional logistic regression. Incidentally, these models also shed light on the near-perfect classification phenomenon for functional data, described by \citet{delaigle2012achieving} and further examined for example in the works of \citet{berrendero2018use} or \citet{torrecilla2020optimal}.

A major aim of this work is to motivate these recently-proposed models inside the functional framework, while also providing efficient techniques to apply them in practice. Our main contribution is the proposal of a Bayesian approach to parameter estimation within the aforementioned RKHS models, in which a prior distribution is imposed on the unknown functional parameter to obtain a posterior distribution after seeing the data. Although setting a prior distribution on a function space is generally a hard task, the specific parametric formulation of the RKHS models we propose greatly facilitates this (see Section~\ref{sec:methodology} for details). A similar Bayesian scheme has recently been explored in \citet{grollemund2019bayesian}, albeit not within a RKHS framework.

Another set of techniques extensively studied in this context are variable selection methods, which aim to select the marginals \(\{X(t_j)\}\) of the process that better summarize it according to some optimality criterion. As it happens, some variable selection methods have already been proposed in the RKHS framework \citep[see for example][]{berrendero2019rkhs}, but in general they have their own dedicated algorithms and procedures. As will become apparent in the forthcoming sections, given the nature of our suggested Bayesian model we can easily isolate the marginal posterior distribution corresponding to a finite set of points \(\{t_j\}\), and thus provide a Bayesian-motivated variable selection process along with the other prediction methods that naturally arise within our model. In this way, in addition to making predictions about the input data, we can evaluate exactly which marginals of the functional explanatory variable contain the most relevant information. These points-of-impact selection models for functional predictors have also been considered in the  literature; see \citet{poss2020superconsistent}, \citet{berrendero2016variable} or \citet{ferraty2010most} by way of illustration. Another example of a related strategy is the work of \citet{james2009functional}, in which the authors propose a method to estimate \(\beta(t)\) in such a way that it is exactly zero over some regions in the domain.


\section{Reproducing kernel Hilbert spaces}\label{sec:rkhs}

For a more detailed account, see for example~\citet{berlinet2004reproducing}.

Suppose \(X=X(t)\) is a \(L^2\)-stochastic process with trajectories in \(L^2[0, 1]\), and without loss of generality we can assume \(\E[X(t)]=0\) for all \(t\in[0,1]\). Let us denote by \(K(t, s)= \E[X(t)X(s)]\) the covariance function of the process \(X\), and in what follows suppose that it is continuous. To construct the RKHS \(\Hcal(K)\) associated with the covariance function, we start by defining the functional space \(\Hcal_0(K)\) of all finite linear combinations of evaluations of \(K\), that is,
\begin{equation}\label{eq:h0}
\Hcal_0(K) = \left\{ f: \ f(\cdot) = \sum_{i=1}^p a_i K(t_i, \cdot), \ p \in \N, \ a_i \in \R, \ t_i \in [0, 1] \right\}.
\end{equation}
Note that, as subsets, \(\Hcal_0(K)\subset L^2[0,1]\). However, this space can be endowed with an inner product different from the one induced by \(L^2[0,1]\), namely \(\dotprod{f}{g}_K = \sum_{i, j} a_i b_j K(t_i, s_j)\) for \(f(\cdot)=\sum_i a_i K(t_i, \cdot) \) and \(g(\cdot)=\sum_j b_j K(s_j, \cdot)\).

\begin{proposition} \(\dotprod{\cdot}{\cdot}_K\) is an inner product.
  \begin{proof}
    Symmetry and linearity are clear from the definition, while non-negativity is a direct consequence of \(K\) being a covariance function, and thus positive semidefinite.
  \end{proof}
\end{proposition}


Then, \(\Hcal(K)\) is defined to be the completion of \(\Hcal_0(K)\) under the norm induced by the scalar product \(\dotprod{\cdot}{\cdot}_K\), turning it into a genuine Hilbert space. As it turns out, functions in this space satisfy the so-called \textit{reproducing property}, which can be easily checked from the definitions.

\begin{proposition}
  If \(f\in \Hcal(K)\), then \(\dotprod{K(t, \cdot)}{f}_K = f(t)\) for all \(t \in [0, 1]\).
  \begin{proof}

\end{proof}
\end{proposition}

This is where the name ``reproducing kernel'' comes from. An important consequence is that \(\Hcal(K)\) is a space of genuine functions and not of equivalence classes, since the values of the functions at particular points are in fact relevant, unlike in \(L^2\)-spaces.

\begin{outcomment}
  A kernel defines an RKHS and vice-versa.
\end{outcomment}

Note that he covariance function of \(X\) (sometimes referred to as the \textit{kernel}) plays a crucial role in characterizing the RKHS. An integral operator closely related to this kernel is the so-called covariance operator, namely \(\mathcal Kf(\cdot) = \int_0^1 K(s, \cdot)f(s)\, ds\) for \(f \in L^2[0, 1]\), which is self-adjoint and compact when \(K\) is continuous \citep[e.g.][Th.~4.6.2]{hsing2015theoretical}. It is worth mentioning that this operator provides several alternative definitions of \(\Hcal(K)\); for example, the RKHS can be identified with the image of the square root of the covariance operator, i.e., \(\Hcal(K) = \mathcal K^{1/2}(L^2[0, 1])\), with inner product \(\dotprod{f}{g}_K = \dotprod{\mathcal K^{-1/2}(f)}{\mathcal K^{-1/2}(g)}\). Furthermore, we can think of the norm in \(\Hcal(K)\) as an \(L^2\)-like regularized norm, since this space can be seen as \(\Hcal(K) = \{f \in L^2[0, 1]: \ \sum_j \lambda_j^{-1}\dotprod{f}{\phi_j}^2 < \infty \}\), where \(\lambda_j\) and \(\phi_j\) are the eigenvalues and (orthonormal) eigenfunctions of \(\mathcal K\), respectively. In this case, the corresponding inner product is \(\dotprod{f}{g}_K = \sum_j \lambda_j^{-1}\dotprod{f}{\phi_j}\dotprod{g}{\phi_j}\). Note that, since the spectral theorem for compact operators tells us that the sequence of eigenvalues \(\{\lambda_j\}\) tends to zero \citep[e.g.][Th.~4.2.4]{hsing2015theoretical}, this definition highlights the fact that functions in \(\Hcal(K)\) are smooth, in the sense that their components in an orthonormal basis need to vanish quickly.

Lastly, a particularly useful approach in statistics is to regard \(\Hcal(K)\) as an isometric copy of a well-known space. Specifically, via \textit{Loève's isometry} \citep{loeve1948fonctions} one can establish a congruence \(\Psi_X\) between \(\Hcal(K)\) and the linear span of the process, \(\mathcal L(X)\), in the space of all random variables with finite second moment, \(L^2(\Omega)\) \citep[see Lemma 1.1 in][]{lukic2001stochastic}. This isometry is essentially the completion of the correspondence
  \begin{equation}\label{eq:loeves-isometry}
  \sum_{i=1}^p a_i X(t_i) \longleftrightarrow \sum_{i=1}^p a_i K(t_i, \cdot),
  \end{equation}
and can be formally defined, in terms of its inverse, as \(\Psi^{-1}_X(U)(t) = \E[U X(t)]\) for \(U \in \mathcal L(X)\).
Despite the close connection between the process \(X\) and the space \(\Hcal(K)\), special care must be taken when dealing with concrete realizations of the process, since under rather general conditions the trajectories of \(X\) do not belong to the corresponding RKHS with probability one \citep[see for example][Cor.~7.1]{lukic2001stochastic}. As a consequence, the expression \(\dotprod{x}{f}_K\) is ill-defined and lacks meaning when \(x\) is a realization of \(X\). However, following Parzen's approach in his seminal work \citep[e.g.][Th.~4E]{parzen1961approach}, we can leverage Loève's isometry and identify \(\dotprod{x}{f}_K \) with the image \( \Psi_x(f) := \Psi_X(f)(\omega)\), for \(x=X(\omega)\) and \(f\in \Hcal(K)\). This notation, viewed as a formal extension of the inner product, often proves to be useful and convenient.

%%%%%%%%%%%%%%%%%%%%%%%%%%%%%%%%%%%%%%%%%%%%%%%%%%%%%%%%%%%%%%%%%%%%%%%%
% Copyright (c) 2022 Antonio Coín
%
% This work is licensed under a
% Creative Commons Attribution-ShareAlike 4.0 International License.
%
% You should have received a copy of the license along with this
% work. If not, see <http://creativecommons.org/licenses/by-sa/4.0/>.
%%%%%%%%%%%%%%%%%%%%%%%%%%%%%%%%%%%%%%%%%%%%%%%%%%%%%%%%%%%%%%%%%%%%%%%%

\let\epsilon\varepsilon

%%%%%%%%%%%%%%%%%%%%%%%%%%%%%%%%%%%%%%%%%%%%%%%%%%%%%%%%%%%%%%%%%%%%%%%%
\chapter{Bayesian methodology for RKHS-based functional regression models}\label{ch:bayesian}
%%%%%%%%%%%%%%%%%%%%%%%%%%%%%%%%%%%%%%%%%%%%%%%%%%%%%%%%%%%%%%%%%%%%%%%%

In this chapter we present the precise functional models and Bayesian methodologies explored in this work. The RKHS-based functional models under consideration \citep[see][]{berrendero2018functional, berrendero2019rkhs} are those obtained by substituting the functional parameter \(\beta\in L^2[0,1]\) for \(\alpha \in \Hcal(K)\), replacing also the scalar product \(\dotprod{X}{\beta}\) for \(\dotprod{X}{\alpha}_K\) in the \(L^2\)-models~\eqref{eq:l2-linear-model} and~\eqref{eq:l2-logistic-model}. However, to further simplify things we will follow a parametric approach and suppose that \(\alpha\) is in fact a member of the dense subspace \(\Hcal_0(K)\) defined in~\eqref{eq:h0}, i.e.:
\begin{equation}\label{eq:alpha_h0}
\alpha(\cdot) = \sum_{j=1}^p \beta_j K(t_j, \cdot), \text{ for some } p \in \N, \ \beta_j \in \R \text{ and } t_j \in [0,1].
\end{equation}
Moreover, as we said before, with a slight abuse of notation we will understand the expression \(\dotprod{x}{\alpha}_K\) as \(\Psi_x(\alpha)\), where \(x=X(\omega)\) and \(\Psi_x\) is Loève's isometry. Hence, taking into account that \(\Psi_X(K(t, \cdot)) = X(t)\) by definition (see~\eqref{eq:loeves-isometry}), we can write \(\dotprod{x}{\alpha}_K \equiv \sum_{j=1}^p \beta_j x(t_j)\) when \(\alpha\) is as in~\eqref{eq:alpha_h0}.

In this way we get a simpler, finite-dimensional approximation of the functional RKHS model, which we argue reduces the overall complexity of the model while still capturing most of the relevant information. When it comes to parameter estimation, a direct optimization of some loss function would probably require a tailored algorithm that took into account the continuous nature of the times \(t_j\). Indeed, such an idea is explored in \citet{berrendero2018functional} for the logistic regression case, where the authors propose a ``greedy max-max'' method reminiscent of the EM algorithm that alternates between estimating the coefficients and the time instants through a maximum likelihood approach.

At this point we propose to follow a Bayesian approach to estimate the parameters of the model, which we believe is in line with the idea of simplicity we pursue, and also introduces an additional layer of flexibility in the model. In this way, we can include problem-specific information in the model through the use of prior distributions, and on top of that, this method works almost unaltered for both linear and logistic regression models. The general idea will be to impose a prior distribution on the functional parameter to eventually derive a posterior model after incorporating the available sample information. In view of~\eqref{eq:alpha_h0}, to set a prior distribution on the unknown function \(\alpha\) (that is, a prior distribution on the functional space \(\Hcal_0(K)\)) it suffices to first consider a discrete distribution on \(p\), and then impose \(p\)-dimensional continuous prior distributions on the coefficients \(\beta_j\) and the times \(t_j\) given \(p\). Thanks to this parametric approach, the challenging task of setting a prior distribution on a space of functions is considerably simplified, while simultaneously not constraining the model to any specific distribution (in contrast to, for instance, Gaussian process regression methods). Moreover, note that starting from a probability distribution \(\P_0\) on \(\Hcal_0(K)\) we can obtain a probability distribution \(\P\) on \(\Hcal(K)\) merely by defining \(\P(B) = \P_0(B\cap \Hcal_0(K))\) for all Borel sets \(B\). Consequently, our simplifying assumption on \(\alpha\) is not very restrictive, since any prior distribution on \(\Hcal_0(K)\) can be directly extended to a prior distribution on \(\Hcal(K)\).

However, after some intial experimentation we found that, for practical and computational reasons, the value of \(p\), the dimensionality of the model, is best fixed beforehand in a suitable way (see Chapter~\ref{ch:model-choice} for details). Thus, we will regard only the \(\beta_j\) and \(t_j\) as free parameters, and search for our functional parameter in the space
\begin{equation}\label{eq:h0p}
\Hcal_{0,p}(K)=\left\{ \sum_{j=1}^p \beta_j K(t_j, \cdot): \ \beta_j \in \R, \ t_j \in [0, 1]\right\}.
\end{equation}
Even though we actually work on \(\Hcal_{0,p}(K)\), the discrete parameter \(p\) can still be selected in several meaningful ways that make use of the available data, and the set of feasible values is not very large in practice. Moreover, we could think of this approach as imposing a degenerate prior distribution on \(p\), so it is in a way a particular case of the more general model discussed above.

In any case, after selecting a suitable prior distribution \(\pi(\theta)\) for the finite-dimensional parameter vector \(\theta\) (which will be specified shortly), we can resort to Bayes' theorem to perform the inference step, which in the case of i.i.d. samples amounts to
\begin{equation}\label{eq:bayes-theorem}
  \pi(\theta \mid \D_n) \propto \left( \prod_{i=1}^n \pi(Y_i\mid X_i, \theta) \right)\pi(\theta).
\end{equation}
In Sections~\ref{sec:rkhs-linear-model} and~\ref{sec:rkhs-logistic-model} we proceed to specify the parameter spaces, prior distributions and concrete models for \(\pi(Y | X,\theta)\) considered in the case of functional linear regression and functional logistic regression, respectively. Even though in~\eqref{eq:bayes-theorem} we have omitted the possibly intractable integral related to the normalizing constant, sampling from the (approximate) posterior distribution can still be accomplished via MCMC methods (see Chapter~\ref{ch:model-choice} for implementation details).

\section{Functional linear regression}\label{sec:rkhs-linear-model}

In the case of functional linear regression, the simplified RKHS model considered in this work is given by
\begin{equation}\label{eq:rkhs-model-linear}
  Y = \alpha_0 + \Psi_X(\alpha) + \epsilon = \alpha_0 + \sum_{j=1}^p \beta_j X(t_j) + \epsilon,
\end{equation}
where \(\alpha(\cdot) = \sum_{j=1}^p \beta_j K(t_j, \cdot) \in \Hcal_{0,p}(K)\), \(\alpha_0\in\R\), and \(\epsilon \sim \mathcal N(0, \sigma^2)\) is an error term independent from \(X\). This model is essentially a finite-dimensional approximation from a functional perspective to the more general RKHS model that assumes \(\alpha \in \Hcal(K)\), proposed in~\citet{berrendero2019rkhs}.

When \(p\) is fixed, the parameter space becomes \(\Theta_p = \R^p \times [0, 1]^p \times \R \times \R^+\), and in the sequel a generic element of this \((2p+2)\)-dimensional space will be denoted by \(\theta = (\beta_1,\dots, \beta_p, t_1,\dots, t_p, \alpha_0, \sigma^2) \equiv (b, \tau, \alpha_0, \sigma^2)\). Before proceeding any further, observe that we can rewrite model~\eqref{eq:rkhs-model-linear} in a more explicit and practical fashion in terms of the available sample information in \(\D_n\). For \(\theta \in \Theta_p\), the reinterpreted model assumes the form
\begin{equation}\label{eq:rkhs-model-linear-2}
  Y_i \mid X_i, \theta \ \stackrel{\text{i.i.d.}}{\sim} \mathcal N\left(\alpha_0 + \sum_{j=1}^p \beta_j X_i(t_j), \ \sigma^2\right), \quad i =1,\dots, n.
\end{equation}

It is worth mentioning that the model remains linear in the sense that it fundamentally involves random variables belonging to the linear span of the process \(X\) in \(L^2(\Omega)\). Also, note that given the time instants \(t_j\), the model becomes a multiple linear model with the \(X(t_j)\) as scalar covariates. As a matter of fact, this RKHS model is particularly suited as a basis for variable selection methods, and also entails the classical \(L^2\)-model~\eqref{eq:l2-linear-model} under certain conditions \citep[see][Sec.~3]{berrendero2020general}. In addition, this model could be easily extended to the case of several covariates via an expression of type \(Y=\alpha_0 + \Psi_{X^{1}}(\alpha_1) + \cdots + \Psi_{X^{q}}(\alpha_q) + \epsilon\). In that case, as argued in \citet{grollemund2019bayesian} for a similar situation, if we were to set a prior distribution on all the parameters involved, we could recover the full posterior by looking alternately at the posterior distribution of each covariate conditional on the rest of them.

\subsection*{The Bayesian approach: prior and posterior}

The prior distribution suggested for the parameter vector \(\theta \in \Theta_p\) is given by
\begin{align}\label{eq:prior-linear}
  \begin{split}
  \pi(\alpha_0, \sigma^2)              & \propto 1/\sigma^2,                                                     \\
  \tau                     & \sim \U([0, 1]^p),                                              \\
  b\mid \tau, \sigma^2 & \sim \mathcal N_p(b_0, g\sigma^2{\underbrace{\left(\X_\tau' \X_\tau + \eta I\right)}_{G_\tau}}^{-1}),
\end{split}
\end{align}
where \(I\) is the identity matrix, \(\X_\tau\) is the data matrix \((X_i(t_j))_{i,j}\), and \(b_0\in \R^p, \ g \in \R\) and \(\eta \in \R^+\) are hyperparameters of the model. On the one hand, note the use of a joint prior distribution on \(\alpha_0\) and \(\sigma^2\), which is a widely used non-informative prior known in the standard linear regression setting as Jeffrey's prior \citep{jeffreys1946invariant}. In any event, the estimation of \(\alpha_0=\E[Y]\) is straightforward, so it could have been left out of the model altogether. On the other hand, the prior on \(b\) is a slight modification of the well-known Zellner's g-prior \citep{zellner1986assessing}, in which a regularizing term is added to avoid ill-conditioning problems in the Gram matrix, obtaining a ridge-like Zellner prior controlled by the tuning parameter \(\eta\) \citep{baragatti2012study}. All in all, with a slight abuse of notation the proposed prior distribution becomes \(\pi(\theta) = \pi(b| \tau, \sigma^2)\pi(\tau)\pi(\alpha_0, \sigma^2)\).

As for the posterior distribution, we only compute a function proportional to its log-density, since that is all that is needed for a MCMC algorithm to work. A standard algebraic manipulation in~\eqref{eq:bayes-theorem} yields the following result.

\begin{proposition}
Under the linear model~\eqref{eq:rkhs-model-linear-2}, the prior distribution implied in~\eqref{eq:prior-linear} produces the log-posterior distribution
\begin{align*}
\log \pi(\theta \mid \D_n) \propto {} & \frac{1}{2\sigma^2}\left(\|\symbf Y- \alpha_0\symbf{1} - \X_\tau b\|^2 + \frac{1}{g}(b - b_0)'G_\tau(b - b_0) \right)\\
& + (p+n+2)\log\sigma - \frac{1}{2}\log |G_\tau|,
\end{align*}
where \(\symbf Y=(Y_1,\dots,Y_n)'\) and \(\symbf{1}\) is an \(n\)-dimensional vector of ones.
\end{proposition}

\subsection*{Making predictions}

In order to generate predictions, let us recall that when performing the empirical posterior approximation, on each of the \(M\) steps of the iterative MCMC algorithm we get an approximate sample \(\theta^{(m)*}=(b^{(m)*}, \tau^{(m)*}, \alpha_0^{(m)*}, (\sigma^2)^{(m)*})\) of the posterior distribution \(\pi(\theta| \D_n)\). Assuming now a previously unseen test set \(\D'_{n'}\) in the same conditions as \(\D_n\), we propose to construct three different kinds of predictors based on the MCMC samples, each of them following a different strategy.

  \paragraph{Summarize-then-predict.} If we consider a point-estimate statistic \(T\) that acts as a summary of the marginal posterior distributions, we can get the corresponding estimates \(\hat{\theta}=(\hat b, \hat \tau, \hat{\alpha}_0, \hat{\sigma}^2) = T\{\theta^{(m)*}\} \equiv (T\{b^{(m)*}\}, T\{\tau^{(m)*}\}, T\{\alpha_0^{(m)*}\}, T\{(\sigma^2)^{(m)*}\})\), and then predict the responses in the usual way following model~\eqref{eq:rkhs-model-linear}, i.e.:
  \begin{equation}\label{eq:summarize-predict-linear}
    \hat Y_i = \hat{\alpha}_0 + \sum_{j=1}^p \hat{\beta}_j X_i(\hat{t}_j), \quad i=1,\dots, n'.
  \end{equation}
  Note that in this case the variance \(\sigma^2\) is treated as a nuisance parameter. Although it contributes to measure the uncertainty in the approximations, its estimates are discarded in the final prediction.

  \paragraph{Predict-then-summarize.} Alternatively, we can  look at the approximate posterior distribution as a whole and compute the predictive distribution of the simulated responses at each step of the chain following model~\eqref{eq:rkhs-model-linear-2}:
  \begin{equation}\label{eq:sampled-response-vector}
  \symbf Y^{(m)*} := \left\{Y_i^{(m)*} \equiv Y_i \mid X_i, \theta^{(m)*}:\ i=1,\dots,n'\right\}, \quad m=1,\dots,M.
  \end{equation}
  Then, we can take the mean of all such simulated responses as a proxy for each response variable, that is,
  \[
  \hat Y_i = \frac{1}{M}\sum_{m=1}^M Y_i^{(m)*}, \quad i=1,\dots,n'.
  \]
  This method differs from the previous one in that it takes into account the full approximate posterior distribution instead of summarizing it directly.

  \paragraph{Variable selection.} Lastly, we can focus only on the marginal posterior distribution of \(\tau|\D_n\) and select \(p\) time instants using a point-estimate statistic \(T\) as in our first strategy, but discarding the rest of the parameters. Specifically, we can consider the times \(\hat t_j = T\{t_j^{(m)*}\}\) and reduce the original data set to just the \(n\times p\) real matrix given by \(\{X_i(\hat t_j): i=1, \dots,n, \ j=1,\dots,p\}\). After this variable selection has been carried out, we can tackle the problem using a finite-dimensional linear regression model and apply any of the well-known prediction algorithms suited for this situation.\\

Note that these predictors can be obtained all at once after only one round of training (that is, an individual MCMC run to approximate the posterior distribution). As a consequence, what we have in practice is a single algorithm that can produce multiple predictors at the same computational cost, so that any of them can be chosen (or even switched back and forth) depending on the particularities of the problem at hand. Moreover, one could even contemplate an \textit{ensemble model} in which some kind of aggregation of several of the available prediction methods is performed to produce a final result. Also, observe that the choice of a specific point estimator to summarize the posterior distribution results in a veiled assumption of an underlying loss function between the estimated and real parameters. In general, the mean is more sensitive to outliers and the median is more robust, but the latter assumes an \(L^1\)-type loss function while the former implicitly optimizes an \(L^2\) loss. On the other hand, the mode is also a good candidate because it represents the point of highest probability density. At any rate, these decisions are strongly dependent on several factors such as the skewness or the number of modes in the resulting posterior distribution, and thus should be made on a case-by-case basis.

\section{Functional logistic regression}\label{sec:rkhs-logistic-model}

In the case of functional logistic regression, we can regard the binary response variable \(Y\in\{0, 1\}\) as a Bernoulli random variable given the regressor \(X=x \in L^2[0, 1]\), and as suppose that \(\log\left(p(x)/(1-p(x))\right)\) is linear in \(x\), where \(p(x)=\P(Y=1| X=x)\). Then, following the approach  suggested by \citet{berrendero2018functional}, a RKHS model is given  by the  equation
\begin{equation}\label{eq:rkhs-model-logistic}
  \P(Y=1 \mid X) = \frac{1}{1 + \exp\{-\alpha_0 - \Psi_X(\alpha)\}}, \quad \alpha_0 \in \R, \ \alpha \in \Hcal_{0,p}(K).
\end{equation}

Indeed, note that this can be seen as a finite-dimensional approximation (but, still, with a functional interpretation) to the general RKHS functional logistic model proposed by these authors, which can be obtained by replacing \(\Hcal_{0,p}(K)\) with the whole RKHS space \(\Hcal(K)\). Now, if we aim at a classification problem, our strategy will be similar to that followed in the functional linear model: after incorporating the sample information, we can rewrite~\eqref{eq:rkhs-model-logistic} as
\begin{equation}\label{eq:rkhs-model-logistic-2}
Y_i \mid X_i,\theta \ \stackrel{\text{i.i.d.}}{\sim} \operatorname{Bernoulli}(p_i), \quad i=1,\dots, n,
\end{equation}
with
\begin{equation}\label{eq:rkhs-model-logistic-2-parameter}
  p_i = \P(Y_i=1 \mid X_i,\theta) = \frac{1}{\displaystyle 1 + \exp\left\{-\alpha_0 - \sum_{j=1}^p \beta_j X_i(t_j)\right\}}, \quad i=1,\dots, n,
\end{equation}
where in turn \(\beta_j\in\R\) and \(t_j\in[0, 1]\).

In much the same way as the linear regression model described above, this RKHS-based logistic regression model offers some advantages over the \(L^2\)-model. First and foremost, it has a more straightforward interpretation and allows for a workable Bayesian approach, as we will demonstrate below. Secondly, it can be shown that under mild conditions the general RKHS logistic functional model holds whenever the conditional distributions \(X | Y=0\) and \(X|Y=1\) are homoscedastic Gaussian processes \citep[see Theorem 1 in][]{berrendero2018functional}; this provides a sound theoretical motivation for the reduced model. Furthermore, a maximum likelihood approach for parameter estimation (although not considered here) is possible as well. Indeed, the use of a finite-dimensional approximation  mitigates the problem of non-existence of the MLE in the functional case. However, let us recall that even in finite-dimensional settings there are cases of quasi-complete separation in which the MLE does not exist \citep{albert1984existence}, though this issue can be circumvented using, for example, Firth's corrected estimator \citep{firth1993bias}.

\subsection*{The Bayesian approach: prior and posterior}

As far as prior distributions go, we propose to use the same ones as we did in~\eqref{eq:prior-linear} for the linear regression model. However, in this case the nuisance parameter \(\sigma^2\) only appears as part of the hierarchical prior distribution, and not in the final model. The posterior distribution is then derived after a routine calculation.

\begin{proposition}
Under the logistic model~\eqref{eq:rkhs-model-logistic-2}, the prior distribution implied in~\eqref{eq:prior-linear} produces the log-posterior distribution
\begin{align*}
  \log \pi(\theta \mid \D_n) \propto {} & \sum_{i=1}^n \left[ \left(\alpha_0 + \Psi_{X_i}(\alpha)\right)Y_i - \log\left(1 + \exp\left\{\alpha_0 + \Psi_{X_i}(\alpha)\right\}\right)\right]\\
  \quad &+ \frac{1}{2}\log |G_\tau| - (p+2)\log \sigma -\frac{1}{2g\sigma^2} (b - b_0)'G_\tau(b - b_0).
\end{align*}
Remember that \(\Psi_{X_i}(\alpha) = \sum_{j=1}^p \beta_j X_i(t_j)\).
\end{proposition}

\subsection*{Making predictions}

Bear in mind that in this case we are essentially approximating the probabilities \(p_i\) in~\eqref{eq:rkhs-model-logistic-2-parameter}, so before producing a response we need to transform the predicted values to a binary output in \(\{0, 1\}\). According to the usual criterion of minimizing the misclassification probability, it is known that the Bayes optimal rule is recovered by predicting \(\hat Y=1\) whenever \(\P(Y=1|X) \geq 1/2\). Nevertheless, for a more general cost function one could consider other criteria that would lead to evaluating whether \(\P(Y=1|X) \geq \gamma\) for some threshold \(\gamma\in[0, 1]\).

With this last strategy in mind, the summarize-then-predict approach on the approximate posterior distribution is analogous to the linear regression case:
\begin{equation}\label{eq:summarize-predict-logistic}
\hat Y_i = \I \left( \left[\displaystyle 1 + \exp\left\{-\hat\alpha_0 - \sum_{j=1}^p \hat\beta_j X_i(\hat t_j)\right\}\right]^{-1} \geq \gamma \right), \quad i=1,\dots,n',
\end{equation}
where \(\I\) is the indicator function (\(\I(P)\) is \(1\) if \(P\) is true and \(0\) otherwise). The hat estimates are obtained once again through a summary statistic \(T\) of the corresponding marginal posterior distributions.

On the other hand, the prediction method that takes into account the entire posterior approximation (that is, the predict-then-summarize approach) is somewhat different now, since there is the question of which response (the Bernoulli variables in~\eqref{eq:rkhs-model-logistic-2} or the raw probabilities in~\eqref{eq:rkhs-model-logistic-2-parameter}) to consider when averaging the posterior samples. Hence, there are primarily two possible outcomes.

  \paragraph{Average sampled probability.} If we choose to average the approximate probabilities \(p_i^{(m)*} = \P(Y_i =1 | X_i,\theta^{(m)*})\) computed following~\eqref{eq:rkhs-model-logistic-2-parameter}, the resulting predictor is
  \[
    \hat Y_i = \I\left(\frac{1}{M} \sum_{m=1}^M p_i^{(m)*} \geq \gamma\right), \quad i=1,\dots,n'.
  \]
  \paragraph{Average sampled response.} Deciding to average the approximate binary responses \(Y_i^{(m)*}\) instead (see~\eqref{eq:sampled-response-vector}) leads to computing the predictions as
  \[
    \hat Y_i = \I\left(\frac{1}{M} \sum_{m=1}^M Y_i^{(m)*} \geq \gamma\right), \quad i=1,\dots,n'.
  \]
  In this case, each \(Y_i^{(m)*}\) follows a Bernoulli distribution of parameter \(p_i^{(m)*}\), for \(m=1,\dots,M\). Note that when \(\gamma=1/2\) this is equivalent to predicting \(Y_i\) from the majority vote of all the \(Y_i^{(m)*}\).\\

Lastly, the variable selection method is essentially the same as in the case of linear regression: we select \(p\) time instants from each trajectory based on a summary of the posterior distribution \(\tau | \D_n\), and then feed the reduced data set to a finite-dimensional binary classification procedure.

%%%%%%%%%%%%%%%%%%%%%%%%%%%%%%%%%%%%%%%%%%%%%%%%%%%%%%%%%%%%%%%%%%%%%%%%
% Copyright (c) 2022 Antonio Coín
%
% This work is licensed under a
% Creative Commons Attribution-ShareAlike 4.0 International License.
%
% You should have received a copy of the license along with this
% work. If not, see <http://creativecommons.org/licenses/by-sa/4.0/>.
%%%%%%%%%%%%%%%%%%%%%%%%%%%%%%%%%%%%%%%%%%%%%%%%%%%%%%%%%%%%%%%%%%%%%%%%

%%%%%%%%%%%%%%%%%%%%%%%%%%%%%%%%%%%%%%%%%%%%%%%%%%%%%%%%%%%%%%%%%%%%%%%%
\chapter{Model choice, implementation and validation}\label{ch:model-choice}
%%%%%%%%%%%%%%%%%%%%%%%%%%%%%%%%%%%%%%%%%%%%%%%%%%%%%%%%%%%%%%%%%%%%%%%%

In this chapter we gather together several remarks on the main choices made during the design and implementation of our model, as well as some examples of validation strategies that attempt to measure the goodness-of-fit of the model given the observed data.

\section{Model specification}

First we describe some problems and modeling decisions that we had to deal with throughout the development of the model.

\subsection*{Label switching}

A well-known issue found in mixture-like models like the ones we propose is \textit{label switching}, which in short refers to the non-identifiability of the components of the model caused by their interchangeability. In our case, this happens because the likelihood is symmetric with respect to the ordering of the parameters \(b\) and \(\tau\), i.e., \(\pi(Y|X,\theta)=\pi(Y|X, \nu(\theta))\) for any permutation \(\nu\) that rearranges the indices \(j=1,\dots, p\). Thus, since the components are arbitrarily ordered, they may be inadvertently exchanged from one iteration to the next in any MCMC algorithm. This can cause nonsensical answers when summarizing the marginal posterior distributions to perform inference, as different labelings might be mixed on each component \citep{stephens2000dealing}.

However, this phenomenon is perhaps surprisingly a condition for the convergence of the MCMC method: as pointed out by many authors \citep[e.g.][]{celeux2000computational}, a lack of switching would indicate that not all modes of the posterior distribution were being explored by the sampler. For this reason, many ad-hoc solutions revolve around post-processing and relabeling the samples to eliminate the switching effect, but they generally do not prevent it from happening in the first place.

The most straightforward solutions consist on imposing an artificial identifiability constraint on the parameters to break the symmetry of their posterior distributions; see \citet{jasra2005markov} and references therein. A common approach that seems to work well is to simply impose an ordering in the parameters in question, which in our case would mean requiring for example that \(\beta_i < \beta_j\) for \(i < j\), or the analogous with \(\tau\). We have implemented a variation of this method described in \citet{simola2021approximate}, which works by post-processing the samples and relabeling the components to satisfy the order constraint mentioned above, choosing either \(b\) or \(\tau\) depending on which set of ordered parameters would produce the largest separation between any two of them (suitably averaged across all iterations of the chains).

This is an area of ongoing research, and thus there are other, more complex relabeling strategies, both deterministic and probabilistic. A summary of several such methods can be found for example in \citet{rodriguez2014label} and \citet{papastamoulis2015label}. In particular, we tested the pivot method proposed by \citet{marin2005bayesian}, in which all samples are aligned to minimize their distance to a reference element (the ``pivot''), which is chosen as the sample that maximizes the posterior density. However, the process of finding the appropriate permutation in each case was time-consuming, and the results were similar, if not worse, than the ones obtained with the simpler order constraints, so the latter were chosen as the default relabeling method of our algorithm (though the pivot method can still be enabled through a dedicated argument in the sampling procedure).

\subsection*{The choice of \(p\)}

One of the key decisions in our Bayesian modeling scheme was whether to consider the number of components \(p\) as a member of the parameter space and integrate it into the model. While theoretically we could impose a prior distribution on \(p\) as well (e.g. a categorical distribution with a fixed maximum value), we found that it would have some unwanted practical implications. For instance, it would make the implementation more complex, since the dimensionality of the parameters \(b\) and \(\tau\) would need to be fixed at a certain maximum value beforehand, but the working value of \(p\) within the MCMC algorithm would vary from one iteration to the next. In this case we would have no immediate way of tracking down which set of parameters is ``active'' at any given time. A simple approach would be to always consider the first \(p\) parameters and ignore the rest, and we did indeed try this technique, but it gave rise to new difficulties and the results obtained were not good. In fact, the label switching issue is accentuated when \(p\) is allowed to vary \citep[c.f.][Sec.~2.3]{grollemund2019bayesian}, and on top of that, the interpretation of, say, the first coefficient \(\beta_1\) in a model with \(3\) components is different than the interpretation of the same coefficient in a model with only \(2\) components.

This inconsistency in the interpretation of the components when the dimensionality of the model increases or decreases can be mitigated using a particular type of MCMC method known as reversible-jump MCMC \citep{green1995reversible}. Theoretically, these algorithms are specifically designed to approximate the posterior distribution in mixture-like models when the number of components is unknown, allowing the underlying dimensionality to change between iterations. However, since they are not yet widely adopted in practice and a reference implementation is not available, we decided against using them in our applications.

Another possibility would be to adapt a purely Bayesian model selection technique to our framework \citep[see][]{piironen2017comparison, gelman2013bayesian}, or even derive some model aggregation methods to combine the posterior distributions obtained for different-sized models. These methods are usually based in computing a quantity known as the \textit{Bayes factor}, which in turn needs the normalizing integral constant we have been trying to avoid all along. In the end, for the sake of simplicity we decided to let \(p\) be an hyperparameter, so that we could use any model selection criteria (e.g. BIC, DIC, cross-validation, \ldots) to select its optimal value. As we will see shortly in Chapter~\ref{ch:experiments}, the experiments carried out indicate that even low values of \(p\) provide sufficient flexibility in most scenarios.

\subsection*{Other hyperparameters}

As for the default values of the rest of hyperparameters in the prior distributions in~\eqref{eq:prior-linear}, several comments are in order:
\begin{itemize}
  \item For the expected value \(b_0\) we propose to use the MLE of \(b\). Although the likelihood function is rather involved, an approximation of the optimal value is enough for our purposes. Our numerical studies suggest that the results are much better with this choice than, say, with a random or null vector.
  \item We found that the parameter \(g\) does not have as much influence on the final result, and the experimentation indicates that \(g=5\) is a good value.
  \item Lastly, we observed that the choice of \(\eta\) can have a considerable impact on the final estimator. That is why, in an effort to normalize its scale, we consider a compound parameter \(\eta = \tilde \eta \lambda_{\max}(\mathcal X_\tau'\mathcal X_\tau)\), where \(\lambda_{\max}(\mathcal X_\tau'\mathcal X_\tau)\) is the largest eigenvalue of the matrix \(\mathcal X_\tau'\mathcal X_\tau\), and \(\tilde\eta > 0\) is the actual tuning parameter (which can be selected for instance by cross-validation strategies). This standardization technique has been used previously in the literature; see for example \citet{grollemund2019bayesian}.
\end{itemize}

\section{MCMC implementation}

The MCMC method chosen for approximating the posterior distribution in our models is the affine-invariant ensemble sampler described in Section~\ref{sec:mcmc}. As mentioned there, we utilize the computational implementation in the Python library \textit{emcee}, which is both reliable and easy to use; it aims to be a general-purpose package that performs well in a wide class of problems. One advantage of this method, apart from the property of affine-invariance, is that it only requires us to specify a few hyperparameters, irrespective of the underlying dimension. This contrasts to, say, the \(\mathcal O(N^2)\) degrees of freedom corresponding to the covariance matrix of an \(N\)-dimensional jump distribution in Metropolis-Hastings.
Furthermore, following \citet{goodman2010ensemble} we can use the \textit{integrated autocorrelation time} to compute an estimate of the effective sample size (i.e. the number of independent samples) after the sampling is done.

After selecting the number of iterations and the number of chains we want, we need to specify the initial points for each of them. As pointed out in \citet{foreman2013emcee}, a good initial choice is a Gaussian ball around a point in \(\Theta_p\) that is expected to have a high probability with respect to the objective distribution. In our implementation we adopt this method, choosing an approximation of the MLE of \(\theta\) as the central point in each case. To perform this approximation we employ the optimization suite of the Python library \textit{scipy}\footnote{\url{https://docs.scipy.org/doc/scipy/}}, and in particular we use the Basin-hopping algorithm \citep{wales1997global}. This is a two-phase stochastic method that combines global steps with local optimization, in the hope of avoiding getting stuck too quickly in local maxima. To reduce the effects of randomness, we run the algorithm a few times and retain the point with the highest likelihood, and to avoid biasing the sampler too much towards the specific point selected, we let a fraction of the initial points be random (within reasonable bounds). This approximation is also used to specify the hyperparameter \(b_0\).

Other less relevant hyperparameters include the burn-in period for the chains, which is the number of initial samples discarded, or the amount of thinning performed, which is the number of consecutive samples discarded to reduce the correlation among them. Another computational decision we made is working with \(\log \sigma\) instead of \(\sigma^2\), so that the domain of this parameter is an unconstrained space, which apparently helps increase the efficiency of the method.

Finally, it is worth mentioning that we experimented with another MCMC framework called \textit{pymc}\footnote{\url{https://docs.pymc.io}}. This is a well-established project aimed at developing a probabilistic programming language for Bayesian modeling with emphasis on MCMC methods \citep{salvatier2016probabilistic}. This package is more complex to use and has a steeper learning curve, but at the same time offers a richer experience and a wide variety of samplers. Apart from standard Metropolis-Hastings techniques, the main algorithm in this library is the \textit{No U-Turn Sampler} (NUTS), a Hamiltonian Monte Carlo method proposed in \citet{hoffman2014no} that performs an automatic adjustment of hyperparameters. However, while the results obtained were similar to that of \textit{emcee}, the execution time of the NUTS sampler was considerably higher (most likely due to parametrization issues with the model). Nevertheless, we decided to include this package as part of the developed code for future use (the Metropolis-Hastings sampler is fully functional and shows comparable performance to \textit{emcee}), although we do not report the corresponding experimental results.

\section{Validation techniques}

To conclude, it is worth mentioning that the Bayesian aspects of our model allow us to perform some model validation checks straight away. For example, we can derive credible intervals for each of the parameters (see Figure~\ref{fig:credible_intervals}), and in the case of linear regression, we can use the sampled values of \(\sigma^2\) as a measure of the uncertainty of the predictions.

\begin{figure}[ht!]
  \centering
    \setlength{\fboxrule}{.5pt}%
  \fbox{\includegraphics[width=.9\textwidth]{credible_intervals}}
  \caption{Example of 94\% credible intervals on the posterior distribution of the impact points.}\label{fig:credible_intervals}
\end{figure}


Moreover, we can perform various visual checks, such as a plot comparing both the observed and posterior predictive distribution of the responses (see Figure~\ref{fig:ppc}). The posterior predictive distribution for a new sample \(x\) is formally defined as
\[
\pi(y\mid x, \mathcal D_n) = \int_{\Theta} \pi(y\mid x, \theta)\pi(\theta\mid \mathcal D_n)\, d\theta,
\]
and in our context it can be approximated by the simulated responses \(\symbf Y^* \equiv \{\symbf Y^{(m)*}\}\), following the notation of~\eqref{eq:sampled-response-vector}. This distribution accounts for the uncertainty about \(\theta\), and we do indeed use it for prediction in our predict-then-summarize approach.

\begin{figure}[ht!]
  \centering
  \setlength{\fboxrule}{.5pt}%
  \fbox{\includegraphics[width=.65\textwidth]{ppc_linear}}
  \caption{Example of posterior predictive graphical checks on a fitted linear regression model. We show a comparison between the observed distribution of the response variable and the posterior predictive distribution consisting on the approximate sampled responses.}\label{fig:ppc}
\end{figure}


In addition, we can calculate the so-called \textit{Bayesian p-values} for several statistics, which are defined as \(P(T(\symbf Y^*)\leq T(\symbf Y)| \symbf Y)\), and are computed by simply measuring the proportion of the \(M\) estimates \(T\{\symbf Y^{(m)*}\}\) that fall below the observed value of the statistic, \(T(\symbf{Y})\). They are expected to be around 0.5 when the model accurately represents the data, and a deviation either way can be indicative of modeling issues; see Chapter 6 of \citet{gelman2013bayesian} for details.

%%%%%%%%%%%%%%%%%%%%%%%%%%%%%%%%%%%%%%%%%%%%%%%%%%%%%%%%%%%%%%%%%%%%%%%%
% Copyright (c) 2022 Antonio Coín
%
% This work is licensed under a
% Creative Commons Attribution-ShareAlike 4.0 International License.
%
% You should have received a copy of the license along with this
% work. If not, see <http://creativecommons.org/licenses/by-sa/4.0/>.
%%%%%%%%%%%%%%%%%%%%%%%%%%%%%%%%%%%%%%%%%%%%%%%%%%%%%%%%%%%%%%%%%%%%%%%%

%%%%%%%%%%%%%%%%%%%%%%%%%%%%%%%%%%%%%%%%%%%%%%%%%%%%%%%%%%%%%%%%%%%%%%%%
\chapter{Experiments}\label{ch:experiments}
%%%%%%%%%%%%%%%%%%%%%%%%%%%%%%%%%%%%%%%%%%%%%%%%%%%%%%%%%%%%%%%%%%%%%%%%



\begin{outcomment}
  Incluir figuras de los datasets, y en general repasar las figuras del notebook.

  Enlazar al github, y comentar que hay demos en jupyter.

  - Note that the use of MCMC algorithms introduces a source of stochasticity in the prediction procedure.

  - Ejemplo en RKHS con p=3 y 2 componentes subyacentes: mostrar todas las gráficas (+ppc).

  - Ejemplo dejando todo constante y aumentando \(p\). Quizás también otro variando el tamaño del grid y/o el número de ejemplos de entrenamiento? Bidimensional en ntrain/ngrid?

  - Sobre experimentos con p-free: Por otro lado, en algunas cadenas el sigma2 es más grande de la cuenta, ya que tiende a "compensar" la falta de flexibilidad del modelo (e.j. si el valor de p es más bajo). Esto hace que los estimadores basados en la media y en el posterior mean sean ligeramente peores (sobre todo el de la media puntual, donde la varianza estimada sí se ve notablemente afectada). Quizás para contrarrestar esto habría que tener un modelo con un $\sigma^2_p$ para cada valor de p?
\end{outcomment}

%%%%%%%%%%%%%%%%%%%%%%%%%%%%%%%%%%%%%%%%%%%%%%%%%%%%%%%%%%%%%%%%%%%%%%%%
% Copyright (c) 2022 Antonio Coín
%
% This work is licensed under a
% Creative Commons Attribution-ShareAlike 4.0 International License.
%
% You should have received a copy of the license along with this
% work. If not, see <http://creativecommons.org/licenses/by-sa/4.0/>.
%%%%%%%%%%%%%%%%%%%%%%%%%%%%%%%%%%%%%%%%%%%%%%%%%%%%%%%%%%%%%%%%%%%%%%%%

%%%%%%%%%%%%%%%%%%%%%%%%%%%%%%%%%%%%%%%%%%%%%%%%%%%%%%%%%%%%%%%%%%%%%%%%
\chapter{Conclusions}\label{ch:conclusions}
%%%%%%%%%%%%%%%%%%%%%%%%%%%%%%%%%%%%%%%%%%%%%%%%%%%%%%%%%%%%%%%%%%%%%%%%

\begin{outcomment}
  Algo que no he comentado en ninguna parte es que el tiempo de ejecución de nuestros modelos es bastante alto (del orden de 30-40 segundos por ejecución) frente al de los algoritmos de comparación (unos pocos segundos en general). Como es algo negativo, no sé muy bien cómo ponerlo, si comentarlo de pasada aquí en las conclusiones, o dedicarle un poco más de espacio en el capítulo de experimentos.\\
\end{outcomment}

\begin{outcomment}
  Algunos comentarios generales:
  \begin{itemize}
    \item Destacar las ventajas de lo bayesiano: flexibilidad, incorporación de información con distintas priors, etc.
    \item Destacar la interpretabilidad y sencillez de nuestro modelo. En general, conseguimos resultados competitivos usando modelos con baja dimensión.
    \item Los resultados en regresión logística parecen en media un poco mejores que en regresión lineal.
    \item No se recomienda usar la media como estadístico de resumen de la distribución a posteriori.
    \item La búsqueda de hiperparámetros en CV no ha sido demasiado grande. Probablemente con más recursos computacionales se podrían conseguir mejores resultados.
  \end{itemize}
\end{outcomment}

\section{Future work}

\begin{outcomment}
  \begin{itemize}
    \item Ahondar en la relación entre RKHS y regresión funcional.
    \item Experimentar con otros algoritmos de MCMC (en particular, reversible-jump MCMC), y con otros paquetes (pymc).
    \item Considerar otras distribuciones a priori.
    \item Derivar algunos resultados de consistencia y/o robustez de la posteriori.
    \item Extender el modelo lineal a un modelo GLM.
  \end{itemize}
\end{outcomment}


  % \part[A good part]{%
  %   A good part\\
  %   %
  %   \vspace{1cm}
  %   %
  %   \begin{minipage}[l]{\textwidth}
  %   %
  %   \textnormal{%
  %     \normalsize
  %     %
  %     \begin{singlespace*}
  %       \onehalfspacing
  %       %
  %       You can also use parts in order to partition your great work
  %       into larger `chunks'. This involves some manual adjustments in
  %       terms of the layout, though.
  %     \end{singlespace*}
  %   }
  %   \end{minipage}
  % }

% This ensures that the subsequent sections are being included as root
% items in the bookmark structure of your PDF reader.
\bookmarksetup{startatroot}
\backmatter

  \printbibliography

\end{document}
