%%%%%%%%%%%%%%%%%%%%%%%%%%%%%%%%%%%%%%%%%%%%%%%%%%%%%%%%%%%%%%%%%%%%%%%%
% Copyright (c) 2022 Antonio Coín
%
% This work is licensed under a
% Creative Commons Attribution-ShareAlike 4.0 International License.
%
% You should have received a copy of the license along with this
% work. If not, see <http://creativecommons.org/licenses/by-sa/4.0/>.
%%%%%%%%%%%%%%%%%%%%%%%%%%%%%%%%%%%%%%%%%%%%%%%%%%%%%%%%%%%%%%%%%%%%%%%%

%%%%%%%%%%%%%%%%%%%%%%%%%%%%%%%%%%%%%%%%%%%%%%%%%%%%%%%%%%%%%%%%%%%%%%%%
\begin{center}
  \begin{Large}
  \textsc{Abstract}
\end{Large}
\end{center}
%%%%%%%%%%%%%%%%%%%%%%%%%%%%%%%%%%%%%%%%%%%%%%%%%%%%%%%%%%%%%%%%%%%%%%%%

\noindent We propose a novel Bayesian approach for functional linear and logistic regression models based on the theory of Reproducing Kernel Hilbert Spaces (RKHS's). These new models build upon the RKHS associated with the covariance function of the underlying stochastic process, and can be viewed as a finite-dimensional approximation to the classical functional regression paradigm. The corresponding functional model (or the functional logistic equation in the case of binary response) is determined by a function living on a dense subspace of the RKHS of interest. By imposing a suitable prior distribution on this space, we can perform data-driven inference via standard Bayes methodology, and the posterior distribution can be estimated through Markov chain Monte Carlo methods. Several prediction strategies derived from this posterior distribution turn out to be competitive against other usual alternatives in both simulated examples and real data sets, including a Bayesian-motivated variable selection procedure.\\

\noindent
\textsc{keywords:} functional data, linear regression, logistic regression, reproducing kernel Hilbert space, Bayesian inference, Markov chain Monte Carlo.

\incomment{add blank page}
