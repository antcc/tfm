%%%%%%%%%%%%%%%%%%%%%%%%%%%%%%%%%%%%%%%%%%%%%%%%%%%%%%%%%%%%%%%%%%%%%%%%
% Copyright (c) 2022 Antonio Coín
%
% This work is licensed under a
% Creative Commons Attribution-ShareAlike 4.0 International License.
%
% You should have received a copy of the license along with this
% work. If not, see <http://creativecommons.org/licenses/by-sa/4.0/>.
%%%%%%%%%%%%%%%%%%%%%%%%%%%%%%%%%%%%%%%%%%%%%%%%%%%%%%%%%%%%%%%%%%%%%%%%

%%%%%%%%%%%%%%%%%%%%%%%%%%%%%%%%%%%%%%%%%%%%%%%%%%%%%%%%%%%%%%%%%%%%%%%%
\chapter{On the code developed}\label{ch:code}
%%%%%%%%%%%%%%%%%%%%%%%%%%%%%%%%%%%%%%%%%%%%%%%%%%%%%%%%%%%%%%%%%%%%%%%%

\begin{outcomment}
  Aquí irá un pequeño resumen de la estructura del código y los archivos, y un mini-tutorial de cómo usarlo para reproducir los experimentos. No debería ocupar más de una página.

  \begin{itemize}
    \item  Comentar cómo usar el código, las versiones de los paquetes, y los notebooks.
    \item Contar que son clases a la sklearn y compatibles con ska-fda. Enlazar al github.
    \item Se usan 4 cores.
    \item El proceso es reproducible (semilla 2022).
  \end{itemize}
\end{outcomment}
