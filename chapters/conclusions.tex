%%%%%%%%%%%%%%%%%%%%%%%%%%%%%%%%%%%%%%%%%%%%%%%%%%%%%%%%%%%%%%%%%%%%%%%%
% Copyright (c) 2022 Antonio Coín
%
% This work is licensed under a
% Creative Commons Attribution-ShareAlike 4.0 International License.
%
% You should have received a copy of the license along with this
% work. If not, see <http://creativecommons.org/licenses/by-sa/4.0/>.
%%%%%%%%%%%%%%%%%%%%%%%%%%%%%%%%%%%%%%%%%%%%%%%%%%%%%%%%%%%%%%%%%%%%%%%%

%%%%%%%%%%%%%%%%%%%%%%%%%%%%%%%%%%%%%%%%%%%%%%%%%%%%%%%%%%%%%%%%%%%%%%%%
\chapter{Conclusions}\label{ch:conclusions}
%%%%%%%%%%%%%%%%%%%%%%%%%%%%%%%%%%%%%%%%%%%%%%%%%%%%%%%%%%%%%%%%%%%%%%%%

\begin{outcomment}
  - Destacar también las ventajas de lo bayesiano: flexibilidad, incorporación de información con distintas priors, etc. En la introducción, conclusiones, experimentos, ..
- Destacar interpretabilidad en los experimentos: e.g. un valor de beta=3.4 se puede interpretar fácilmente en un caso concreto. En general, nuestros modelos tienen menor dimensión.
- Logistic un poco mejor que linear?
- Tiempo de ejecución alto (característico de los algoritmos MCMC), a cambio mucha sencillez.
- CV ha sido pequeño. Con más parámetros, probablemente mejores resultados. También flexibilidad para cambiar las prioris.
- We do not advise to use the mean as a summary statistic.
\end{outcomment}

\section{Future work}


\begin{outcomment}
  - Mejorar pymc.
  - consistencia de la posteriori, robustez de la priori
  - Otras prioris

\end{outcomment}
