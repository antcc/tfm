%%%%%%%%%%%%%%%%%%%%%%%%%%%%%%%%%%%%%%%%%%%%%%%%%%%%%%%%%%%%%%%%%%%%%%%%
% Copyright (c) 2022 Antonio Coín
%
% This work is licensed under a
% Creative Commons Attribution-ShareAlike 4.0 International License.
%
% You should have received a copy of the license along with this
% work. If not, see <http://creativecommons.org/licenses/by-sa/4.0/>.
%%%%%%%%%%%%%%%%%%%%%%%%%%%%%%%%%%%%%%%%%%%%%%%%%%%%%%%%%%%%%%%%%%%%%%%%

%%%%%%%%%%%%%%%%%%%%%%%%%%%%%%%%%%%%%%%%%%%%%%%%%%%%%%%%%%%%%%%%%%%%%%%%
\chapter{Conclusions}\label{ch:conclusions}
%%%%%%%%%%%%%%%%%%%%%%%%%%%%%%%%%%%%%%%%%%%%%%%%%%%%%%%%%%%%%%%%%%%%%%%%

\begin{outcomment}
  Algunos comentarios generales:
  \begin{itemize}
    \item Destacar las ventajas de lo bayesiano: flexibilidad, incorporación de información con distintas priors, etc. Computationally feasible model.
    \item Destacar la interpretabilidad y sencillez de nuestro modelo. En general, conseguimos resultados competitivos usando modelos con baja dimensión. Aunque el modelo final es simple, la teoría subyacente no lo es. Modelo simple útil para practitioners. También se presta a post-interpretación (ej. si salen dos modas)
    \item Algo del tiempo de ejecución. Se podría aproximar mejor el error medio con más repeticiones aleatorias (en lugar de 10), pero sería muy costoso computacionalmente.
    \item Los resultados en regresión logística parecen en media un poco mejores que en regresión lineal.
    \item No se recomienda usar la media como estadístico de resumen de la distribución a posteriori. En general, los que usan la posteriori entera son más estables.
    \item La búsqueda de hiperparámetros en CV no ha sido demasiado grande. Probablemente con más recursos computacionales se podrían conseguir mejores resultados.
  \end{itemize}
\end{outcomment}

\section{Future work}

\begin{outcomment}
  \begin{itemize}
    \item Ahondar en la relación entre RKHS y regresión funcional.
    \item Experimentar con otros algoritmos de MCMC (en particular, reversible-jump MCMC), y con otros paquetes (pymc).
    \item Considerar otras distribuciones a priori.
    \item Derivar algunos resultados de consistencia y/o robustez de la posteriori.
    \item Extender el modelo lineal a un modelo GLM.
  \end{itemize}
\end{outcomment}
